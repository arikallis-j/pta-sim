%Объявление стилей документа
\documentclass[twocolumn, 10pt]{article}

\setlength{\columnsep}{1cm}

%Русская кодировка и шрифт
\usepackage[english, russian]{babel}
\usepackage[T2A]{fontenc}
\usepackage[utf8]{inputenc}

%Пакет для использования теорем
\usepackage{amsthm}
%Установка стиля для теоремы
\theoremstyle{plain}

%Создание нумерованного окружения для теорем
\newtheorem{theorem}{Теорема}

%Создание ненумерованного окружения для лемм
\newtheorem*{lemma}{Лемма}

%Пакеты Американского Математического Общества, которые никогда не будут лишними:
\usepackage{amsmath}
\usepackage{amsfonts}
\usepackage{amssymb}
\usepackage{mathtools}
\author{Черносов Вячеслав}
\title{Моделирование гравитационных волн}

\begin{document}
\maketitle
\section{Описание модели PTA}
\subsection{Симулированные данные}
В качестве внешнего параметра нашей модели положим спектр $H(f)$. Тогда в Фурье-пространстве ход гравитационной волны равен:
$$\tilde{h}(f) = \sqrt{H(f)},$$
а в физическом пространстве: 
$$\mathbf{h}(t) = \int e^{i 2 \pi f t} \cdot \tilde{h}(f) \cdot \mathrm{d} f.$$
Разность хода волны, принимаемая на Земле будет равна:
$$\Delta \mathbf{h}(t, \hat{\Omega}) = \mathbf{h}(t) - \mathbf{h}(t - L (1 + \hat{\Omega} \cdot \hat{p})).$$
Тогда красное смещение, фиксирующееся на Земле будет равно:
$$\mathbf{Z}(t, \hat{\Omega}) = \Delta \mathbf{h}(t, \hat{\Omega}) \cdot \mathbf{F}^{*}(\hat{\Omega}),$$
где:
$$\mathbf{F}(\hat{\Omega}) = \frac{1}{2} \frac{\hat{p}^i \hat{p}^j}{1 + \hat{\Omega} \cdot  \hat{p}} \cdot \mathbf{e}_{ij}(\hat{\Omega}).$$
Проинтегруем по сфере:
$$\mathbf{Z}(t) = \int   \Delta \mathbf{h}(t, \hat{\Omega}) \cdot \mathbf{F}^{*}(\hat{\Omega}) \cdot \mathrm{d} \hat{\Omega}.$$
Действительная часть сигнала будет равна:
$$Z(t) = \mathfrak{R}\left[\mathbf{Z}(t)\right]$$

\subsection{Обработка данных}
Посчитаем корреляцию сигналов от двух пульсаров:
$$ r_{ab} (\tau) = \int Z_a(t) \cdot Z_b(t - \tau) \cdot \mathrm{d} t $$
Корреляция для $\tau = 0$ описывается как:
$$r_{ab} (0) \approx 4 \pi h^2 \frac{1}{4 \pi} \int  \mathbf{F}_a(\hat{\Omega})  \cdot \mathbf{F}_b(\hat{\Omega}) \cdot \mathrm{d} \hat{\Omega},$$
где:
$$h^2 = \int H(f) \cdot \mathrm{d} f .$$
Тогда функция $\mu_{ab}(\gamma)$ из наблюдательных данных получается как:
$$\mu_{ab}(\gamma) = \frac{1}{4 \pi h^2} \int Z_a(t) \cdot Z_b(t) \cdot \mathrm{d} t$$
\subsection{Форма спектра}
Рассмотрим две основных формы спектра:
\begin{enumerate}
    \item  Сигнал на одной частоте $f_0$ будет описываться как:
    $$H(f) = A \cdot \delta(f - f_0)$$
    \item  Сигнал со степенным спектром $\alpha$ будет описываться как:
    $$H(f) = A \cdot f^{\alpha}$$
\end{enumerate} 
\end{document}